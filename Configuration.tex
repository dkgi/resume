\usepackage{xspace}
\usepackage{url}
\usepackage{etoolbox}

\newcommand{\correspondence}{% CORRESPONDENCE INFORMATION
  \address{Dominik Gabi\\ Niderenweg 18\\ CH-9043 Trogen\\ +41 79 789 41 24\\[12pt]}
  \address{\url{https://github.com/dkgi}\\ \url{dkgiwork@gmail.com}}
}

% FORMALITIES
\newcommand{\contact}{COMPANY CONTACT\xspace}
\newcommand{\company}{COMPANY NAME\xspace}
\renewcommand{\position}{Entry Level Software Engineer}
\newcommand{\objective}{\position\ at \company}
\newcommand{\duration}{DURATION\xspace}
	
% Personal correspondence?
\newtoggle{contact}
\toggletrue{contact}

% Apply through cds international?
\newtoggle{cds}
\togglefalse{cds}				


\newcommand{\companyspecific}{% COMPANY SPECIFIC
  COMPANY SPECIFIC.\xspace
}

\newcommand{\applicationspecific}{% APPLICATION SPECIFIC
  APPLICATION SPECIFIC.\xspace
}

% CURRICULUM VITAE

% Use with \education{title}{year}{description}
\newcommand{\education}[3]{\textbf{#1} \hfill #2\\#3\\[12pt]}

\newcommand{\educations}{% EDUCATIONS
		\education{Master of Science in Computer Science}{2011}{
			Federal Institute of Technology (ETH), Z\"urich\\
			Thesis: Disjunction on Demand\\
			Advisers: Prof. Dr. Peter M\"uller, Dr. Pietro Ferrara}
		\education{Bachelor of Science in Computer Science}{2010}{
			Federal Institute of Technology (ETH), Z\"urich}
		\education{High School}{2005}{Kantonsschule, Trogen}
}

% Use with \experience{title}{description}{url}
\newcommand{\experience}[3]{\textbf{#1}\\#2\\\url{#3}\\[12pt]}

\newcommand{\experiences}{% EXPERIENCES
		\experience{Master's Thesis Disjunction on Demand}{My master's thesis introduced the \emph{Trace Partitioning Abstract Domain} into the static analyzer \emph{Sample} that is currently being developed at ETH. This abstract domain allows the analysis to distinguish in a tractable way between specific traces without having to worry about an explosion in complexity.}{http://www.pm.inf.ethz.ch/education/theses/master_completed}
		\experience{Software Verification Course Project}{I implemented and proved correct a recursive algorithm solving the maximum common sublist problem. The implementation was written in \texttt{C} and proved using the \texttt{VeriFast} program verifier.}{https://github.com/dkgi/sublist-proof}
		\experience{Concepts of Concurrent Computation Course Project}{In this project I implemented a general framework for concurrent divide and conquer algorithms using the \texttt{SCOOP} framework and demonstrated its functionality with implementations of merge-sort, finding the longest word in a text and finding the two closest points in a two dimensional plane.}{https://github.com/dkgi/divide-and-conquer-scoop}
}

\newcommand{\interests}{% INTERESTS
		Software Engineering, Software Architecture, Software Verification, Concurrency, Computer Graphics, Simulation, Functional Programming, Scala, Java, C++
}

\newcommand{\competentskills}{% COMPETENT SKILLS
	Scala, Java, C/C++, Matlab
}

\newcommand{\familiarskills}{% FAMILIAR SKILLS
	VeriFast, PROMELA/SPIN, C\#, Eiffel
}

\newcommand{\languages}{% LANGUAGES
	Fluent in English and German
}

\newcommand{\reference}[3]{\textbf{#1}\\#2\\[2pt]\url{#3}\\}

\newcommand{\references}{% REFERENCES
		\reference{Dr. Pietro Ferrara}{Researcher, Federal Institute of Technology, Z\"urich}{pietro.ferrara@inf.ethz.ch}
}
