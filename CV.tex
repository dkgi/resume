\documentclass[margin]{res}

\usepackage{xspace}
\usepackage{url}
\usepackage{etoolbox}

\newcommand{\correspondence}{% CORRESPONDENCE INFORMATION
  \include{Personal}  % Not commited
  %\address{Your address here\\[12pt]}
  %\address{Your email and url here\\}
}

\newcommand{\company}{COMPANY NAME\xspace}
\newcommand{\duration}{DURATION\xspace}
\newcommand{\companyspecific}{% COMPANY SPECIFIC
  COMPANY SPECIFIC.
}

\newcommand{\applicationspecific}{% APPLICATION SPECIFIC
  APPLICATION SPECIFIC.
}

\newtoggle{cds}
\toggletrue{cds}



\begin{document}

\name{CV\\[12pt]}
\correspondence

\begin{resume}

	% OBJECTIVE

	\begin{section}{Objective}

		Internship at \company

	\end{section}

	% EDUCATION

	\begin{section}{Education}

		\textbf{Master of Science in Computer Science} \hfill 2011\\
		Federal Institute of Technology (ETH), Z\"urich\\
		Thesis: Disjunction on Demand\\
		Advisers: Prof. Dr. Peter M\"uller, Dr. Pietro Ferrara\\[12pt]
		\textbf{Bachelor of Science in Computer Science} \hfill 2010\\
		Federal Institute of Technology (ETH), Z\"urich\\[12pt]
		\textbf{High School} \hfill 2005\\
		Kantonsschule, Trogen

	\end{section}

	% EXPERIENCE

	\begin{section}{Experience}

		\textbf{Master's Thesis Disjunction on Demand}\\
		My master's thesis introduced the \emph{Trace Partitioning Abstract Domain} into the static analyzer \emph{Sample} that is currently being developed at ETH. This abstract domain allows the analysis in a tractable way to dynamically distinguish between states based on how they are reached.\\
		\url{http://www.pm.inf.ethz.ch/education/theses/master_completed}\\[12pt]
		\textbf{Software Verification Course Project}\\
		I implemented and proved correct a recursive algorithm solving the maximum common sublist problem. The implementation was written in \texttt{C} and proved using the \texttt{VeriFast} program verifier.\\[2pt]
		\url{https://github.com/dkgi/sublist-proof}\\[12pt]
		\textbf{Concepts of Concurrent Computation Course Project}\\
		In this project I implemented a general framework for concurrent divide and conquer algorithms using the \texttt{SCOOP} framework and demonstrated its functionality with implementations of merge-sort, finding the longest word in a text and finding the two closest points in a two dimensional plane.\\[2pt]
		\url{https://github.com/dkgi/divide-and-conquer-scoop}

	\end{section}

	% INTERESTS

	\begin{section}{Interests}

		Verification, Formal Methods, Static Analysis, Model Checking, Functional Programming, Scala,
		Parallel Programming, Concurrency, Computer Graphics, Physically-Based Simulation

	\end{section}

	% TECHNICAL SKILLS

	\begin{section}{Technical Skills}

		\textbf{Competent:} Scala, Java, C/C++, Matlab\\
		\textbf{Familiar:} VeriFast, PROMELA/SPIN, C\#, Eiffel

	\end{section}

	% REFERENCES

	\begin{section}{References}

		\textbf{Dr. Pietro Ferrara}\\ 
		Researcher, Federal Institute of Technology, Z\"urich\\[2pt]
		\url{pietro.ferrara@inf.ethz.ch}\\

	\end{section}

\end{resume} 

\end{document}
